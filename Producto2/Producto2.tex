\documentclass[11pt]{article} 
\usepackage [spanish] {babel}
\textwidth=15cm
\usepackage [T1]{fontenc}
\usepackage{graphics}
%%%Define colours and length
\usepackage [latin1]{inputenc}
\begin{document}
 
\title{TRADUCTORES: COMPILADORES E INTERPRETES} 
\author{Ana Magdalena Sotomayor} 

\maketitle 
\section{INTRODUCCION}
Los traductores son programas que nos permiten interactuar con los ordenadores. Son aquellos que  toman como entrada un programa escrito en lenguaje simbolico y comprensible por los usuarios, que se denomina programa o codigo fuente y proporciona como salida otro programa escrito en un lenguaje comprensible por el hardware del ordenador, denominado programa objeto cuyo objetivo es que el ordenador realize el trabajo codificado.

Un compilador traduce completamente un programa fuente, escrito en un lenguaje de alto nivel, a un programa objeto, escrito en lenguaje ensamblador o maquina. El programa objeto resultante es independiente del compilador y puede ser procesado posteriormente sin volver a realizar la traduccion, y toda interaccion con el usuario estara controlada por el sistema operativo. 

La traduccion por un compilador a la que se le llama compilacion, consta de dos etapas:la etapa de analisis del programa fuente y la etapa de sintesis del programa objeto. El analisis del texto fuente implica la realizacion de un analisis del lexico, de la sintaxis y de la semantica. La sintesis del programa objeto conduce a la generacion de codigo y su optimizacion. El compilador informa al usuario de cualquier error que se presente durante el analisis del texto y no crea el programa objeto hasta que los errores se hayan eliminado.

Por otro lado, el interprete permite que un programa fuente escrito en un determinado lenguaje vaya traduciendose y ejecutandose directamente, sentencia a sentencia, por el ordenador. El interprete capta una sentencia fuente, la analiza e interpreta, dando lugar a su ejecucion inmediata, no creandose, por tanto, un archivo o programa objeto almacenaje en memoria masiva para posteriores ejecuciones. La ejecucion del programa estara supervisada por el interprete.

\section{TABLA COMPARATIVA PRINCIPALES TRADUCTORES} 
 \begin{tabular}{|p{2.0cm}|p{4.0cm}|p{2.0cm}|p{2.0cm}|p{3.0cm}|}
 \hline
 NOMBRE & PARADIGMA & CREADORES & APARICION & EXTENSIONES \\ \hline
 C & Imperativo, Procedural,Estructurado & Dennis Ritchie & 1972 & .h . c \\ \hline 
C++ & Multiparadigma, orientado a objetos, imperativo, programacion generica. & Bjarne Stroustrup & 1983& .h .hh .hpp .hxx .h++ .cc .cpp .cxx .c++ \\ \hline
Fortran & Programacion de arreglos, programacion modular, orientada a objetos y generica. & John W. Backus & 1957 & .f .for .f90 .f95\\ \hline
Java & Multiparadigma, orientado a objetos, estructurado, imperativo, funional, generico, reflectivo, concurrente. & James Gosling & 1955 & .java .class .jar\\ \hline
Python & Multiparadigma, orientado a objetos, imperativo, reflexivo, funional. & Guido van Rossum & 1991 & .py -pyc .pyd .pyo .pyw\\ \hline
Ruby & Multiparadigma, orientado a objetos, imperativo, funcional, reflectivo. & Yukihiro Matsumoto & 1995 & .rb .rbw\\ \hline

 \hline
\end{tabular}
\section{EJEMPLOS DE LENGUAJE DE COMPILACION/INTERPRETACION} 
\subsection*{C}
\begin{verbatim}#include <iostream>
 
int main()
{
    printf("Hola! Trataré de adivinar un número.\n");
     printf("Piensa un número entre 1 y 10.\n");
    sleep(5)	
     printf("Ahora multiplicalo por 9.\n");
    sleep(5)
    printf("Si el número tiene 2 dígitos, súmalos entre si: Ej. 36->3+6=9. Si tu número tiene un solo dígito, súmale 0.\n");
    sleep(5)
     printf( "Al número resultante súmale 4.\n");
    sleep(10)	
     printf( "Muy bien. El resultrado es :)\n"); 
}\\ \hline
\end{verbatim}
\subsection{C++}
\begin{verbatim}
 #include <iostream>
 
int main()
{
    printf("Hola! Trataré de adivinar un número.\n");
     printf("Piensa un número entre 1 y 10.\n");
    sleep(5)	
     printf("Ahora multiplicalo por 9.\n");
    sleep(5)
    printf("Si el número tiene 2 dígitos, súmalos entre si: Ej. 36->3+6=9. Si tu número tiene un solo dígito, súmale 0.\n");
    sleep(5)
     printf( "Al número resultante súmale 4.\n");
    sleep(10)	
     printf( "Muy bien. El resultrado es :)\n"); 
} 
\end{verbatim}
\subsection{Fortran}
\begin{verbatim}
program Adivina 

  write(*,*) 'Hola! Trataré de adivinar un número.';
     write(*,*) 'Piensa un número entre 1 y 10.\n';
     call sleep(5)	
     write(*,*) 'Ahora multiplicalo por 9.';
     call sleep(5)
    write(*,*) 'Si el número tiene 2 dígitos, súmalos entre si: Ej. 36->3+6=9. Si tu número tiene un solo dígito, súmale 0.';
     call sleep(5)
     write(*,*) 'Al número resultante súmale 4.';
     call sleep(10)	
     write(*,*) 'Muy bien. El resultado es 13 :)'; 
end program Adivina
\end{verbatim}
\subsection{Java}
\begin{verbatim}
public class Enjava {
    public static void main(String[] args) {
    
    System.out.println("Hola! Trataré de adivinar un número.");

    System.out.println("Piensa un número entre 1 y 10.");

 try {
    Thread.sleep(1000);
} catch (InterruptedException e) {

    System.out.println("Ahora multiplicalo por 9.");
}
 try {
    Thread.sleep(1000);
} catch (InterruptedException e)
{
    System.out.println("Si el número tiene 2 dígitos, súmalos entre si: Ej. 36->3+6=9. Si tu número tiene un solo dígito, súmale 0.");
}
 try {
    Thread.sleep(1000);
} catch (InterruptedException e)
{
    System.out.println("Al número resultante súmale 4.");
}
try {
    Thread.sleep(1000);
} catch (InterruptedException e)
{
    System.out.println("Muy bien. El resultado es 13. :)"); 
 }   }

}
\end{verbatim}
\subsection{Python}
\begin{verbatim}
import time
print("Hola! Tratare de adivinar un numero.")
print("Piensa un numero entre 1 y 10.")
time.sleep(5)	
print("Ahora multiplicalo por 9.")
time.sleep(5)
print("Si el numero tiene 2 digitos, sumalos entre si. Si tu numero tiene un solo digito, sumale 0.")
time.sleep(5)
print( "Al numero resultante sumale 4.")
time.sleep(10)	
print( "Muy bien. El resultado es 13. :)") \\ \hline
\subsection{Ruby}
\begin{verbatim}
puts
    puts"Hola! Trataré de adivinar un número." 
     puts"Piensa un número entre 1 y 10." 
    sleep(5)	
     puts"Ahora multiplicalo por 9." 
    sleep(5)
    puts"Si el número tiene 2 dígitos, súmalos entre si: Ej. 36->3+6=9. Si tu número tiene un solo dígito, súmale 0." 
    sleep(5)
     puts "Al número resultante súmale 4." 
    sleep(10) 	
     puts "Muy bien. El resultado es 13. :) " 
\end{verbatim}     

% Nunca debe faltar esta ultima linea.
\end{document}