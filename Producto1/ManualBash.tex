% Ejemplo de documento LaTeX
% Tipo de documento y tamaño de letra
\documentclass[letterpaper,10pt,twoside,onecolumn]{article}
\textwidth=15cm
% Preparando para documento en Español.
% Para documento en Inglés no hay que hacer esto.
\usepackage[spanish]{babel}
\selectlanguage{spanish}
\usepackage[utf8]{inputenc}

% EL titulo, autor y fecha del documento
\title{Manual breve de comandos de Linux}
\author{Ana Magdalena Sotomayor}
\date{29 de Enero del 2015}

% Aqui comienza el cuerpo del documento
\begin{document}
% Construye el título
\maketitle

\section{COMANDOS B\'ASICOS}

\begin{tabular}{|p{2.0cm}|p{6.0cm}|p{4.0cm}|}
\hline
Comando & Descripción & Ejemplo \\
\hline
echo & Se utiliza para desplegar mensajes & \$Shell \\ \hline
pwd & despliega el directorio en el que se trabaja actualmente & pwd
\\ \hline
ls & listar contenido de directorios & ls -a \\ \hline
cd & Nos permite movernos entre directorios & cd .. \\ \hline
file [path]	 &Despliega el tipo de archivo del que estamos preguntando. & File notas
\\ \hline
man &Despliega las hojas de manual del comando que solicitemos & man -k
\\ \hline
mkdir & Nos permite crear un directorio nuevo.& mkdir Nuevo \\ \hline
touch & Crea un archivo en blanco & touch archivo \\ \hline
cp & Copia un archivo  o directorio al path indicado & cp nuevo nuevo2 \\ \hline
mv & Mueve un archivo o directorio al path indicado & mv nuevo2 /Nuevo/Archivo \\ \hline
rm & Remueve o borra un archivo & rmdir Nuevo\\ \hline
vi & Permite editar un archivo & vi nuevo2 \\ \hline
cat & Permite visualizar el contenido de un archivo. & cat nuevo2 \\ \hline
less & Permite ver el contenido de archivos de texto & less nuevo2.txt \\ \hline
Ctrl + C & conjunto de comandos para CANCELAR el comando anterior  & Ctrl + C\\
\hline
\end{tabular} 

% Nunca debe faltar esta última linea.
\end{document}
