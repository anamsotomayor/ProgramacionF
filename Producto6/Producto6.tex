% Tipo de documento y tamaño de letra
\documentclass[letterpaper,10pt,twoside,onecolumn]{article}

\textwidth=12cm
% Preparando para documento en Español.
% Para documento en Inglés no hay que hacer esto.
\usepackage[spanish]{babel}
\usepackage{graphicx}
\selectlanguage{spanish}
\usepackage[utf8]{inputenc}
\begin{document}
 
\title{PRODUCTO 6}

\author{Ana Magdalena Sotomayor} 

\maketitle 
\section{INTRODUCCION}
El lanzamiento de un proyectil puede resultar en dos tipos de movimientos a estudiar, dependiendo de medio en el que se mueve.

Si, para su estudio, determinamos que el proyectil se mueve por un medio sin fricción, entonces se genera un Tiro Parabólico.

El tiro parabólico es un movimiento que resulta de la unión de dos movimientos: El movimiento rectilíneo uniforme (componentes horizontal) y, el movimiento vertical (componente vertical) que se efectúa por la gravedad y el resultado de este movimiento es una parábola. 

\includegraphics[scale=1]{baloncesto.png}

Las formulas para este movimiento son sencillas y dependen del ángulo de tiro y de la velocidad inicial del lanzamiento:
\begin{equation}
Voy = Vo sin \theta
Vox = Vo cos \theta
 \end{equation}

Y la distancia para generar las coordenadas de la trayectoria se saca a partir de las siguientes fórmulas:

$X= Vox*t$
$Y= Voy*t-1/2gt^2$

Por otro lado, si tomamos en cuenta que el medio real en el que se mueve un objeto lanzado como proyectil genera una fuerza de fricción o arrastre, la trayectoria del movimiento difiere.

\includegraphics[scale=1]{cinematica.png}

Todo objeto de masa m que se mueve en un medio, experimenta una fuerza de arrastre FD contraria a la dirección de su movimiento; se expresa de la siguiente manera

$ F_D = 1/_2$ $\rho$ $^2C_DA$

Que depende de una constante $\rho$ definido por la densidad del medio y $C_A$ definida como el coeficiente de fricción por la forma del objeto lanzado.

Por lo anterior, la trayectoria del proyectil se complica, teniendo en cuenta que la fricción genera una desaceleración constantemente, por lo que las formulas para la obtención de las coordenadas requieren de una constante modificación de aceleración y velocidad en x y y.

Aceleración:
$a_x = 2 1 D/m 2 vvx $
$a_y = 2g 2 1 D/m 2 vv$
Velocidad:
$vx + Dvx = vx + axDt$
$ vy + Dvy = vy + ayDt$
Coordenadas
$x+/delta x =  Xo + Vx/delta t+ 1/2ax/delta t^2$
$y+/delta y =  Yo + Vy/delta t+ 1/2ay/Delta t^2$

Se realizó un código en fortran 90 para realizar los calculos necesarios para graficar las trayectorias con objetos a escoger y en tomando como medio el aire a 25 grados centígrados con una atmósfera de presión.

Para ello se crearon módulos de parámetros, subrutinas y diversas herramientas como se verá en los códigos a continuación.
\pagebreak

\section{CÓDIGOS} 
\begin{verbatim}

!Programa para obtener los valores cada décima de segundo
!para graficar un tiro parabólico de un objeto de diferentes formas,
!Sin y con fuerza de arrastre del medio en el que se mueve
!Código por Ana M. Sotomayor
!----------------------------------------------------------------------------|
!Definamos parametros
Module Parametros

    Implicit None
    real, parameter :: pi = 4.0 * atan(1.0)
!Definimos el valor de la densidad de aire a un aprox de 20 grados C con una presion
atmosferica de 1
    real, parameter :: dAire = 1.2
    real, parameter :: g = 9.8
    integer, parameter :: puntos = 3500
    real, parameter :: Delta = 0.010

End Module Parametros

Program Producto6

    Use Parametros
    Implicit None
    real :: Vox, Voy, a_rad !Salida a Subrutinas
    real :: Ttot, Ytot, Xtot !Entrada de Subrutina Sin Arrastre
    real :: Ttotal, Ytotal, Xtotal, A, CD    !Entrada de Subrutina 
    		Con Arrastre
    real :: DiferenciaX, DiferenciaY, DiferenciaT !Internas
    real :: Vo, a_grados
    integer :: i
    real, dimension (0:puntos) :: x,y, ts !De Subrutina sin Arrastre
    Character :: Objeto !De Subrutina con Arrastre
    real, Dimension (0:puntos) :: t, Vx, Vy, Vin, ax, ay, X1, Y1 !De 
								    Subrutina con arrastre

    Write (*,*) "Escriba la velocidad inicial del Objeto"
    Read *,Vo
    write (*,*) "Escriba en grados el angulo de salida"
    Read *,a_grados

!Se convierten grados a radianes
    a_rad = a_grados*pi/180

!Se convierte la velocidad a sus componentes en x y y
    Vox = Vo*cos(a_rad)
    Voy = Vo*sin(a_rad)
!Llamamos a las subrutinas para obtener maximos

Call SinFriccion (Vo, Vox, Voy, ts, i, x, y, Ytot, xtot, Ttot)
Call ConArrastre (Vox, Voy, a_rad, Vo, CD, A, t, Vx, Vy, ax, ay,X1, Y1, 
Xtotal, Ytotal, Ttotal, Objeto)

DiferenciaX = ((xtot-Xtotal)/Xtotal)*100.0
DiferenciaY = ((Ytot-Ytotal)/Ytotal)*100.0
DiferenciaT = ((Ttot-Ttotal)/Ttotal)*100.0

Write (*,*) "Al lanzar un objeto de la forma seleccionada"
Write (*,*) "Con una velocidad incial de", Vo, "m/s"
Write (*,*) "Y un angulo de", a_grados, "grados"
Write (*,*) "Su trayectoria sin Friccion tiene una duracion de", Ttot, "segundos"
Write (*,*) "Y alcanza una altura maxima de", Ytot, "metros"
Write (*,*) "y un alcance de", Xtot, "metros"
Write (*,*) "_-_-_-_-_-_-_-_-_-_-_-_-_-_-_-_-_-_-_-_-_-_-_-_-_-_-_-_-_-_-_-_-_-"
Write (*,*) "Mientras que tomando en cuenta el arrastre del aire"
Write (*,*) "Su tiempo total de vuelo es de", Ttotal, "segundos"
Write (*,*) "Con una altura maxima de", Ytotal, "metros"
Write (*,*) "y un alcance de", Xtotal, "metros"
Write (*,*) "Lo que crea un porcentaje de error de", DiferenciaT, "en el tiempo"
Write (*,*) "de", DiferenciaY,"en la altura maxima y de", DiferenciaX," en el alcance" 

End Program Producto6

!-----------------------------------------------------------------------------
!Subrutina para generar las coordenadas de la trayectoria sin arrastre

Subroutine SinFriccion (Vo, Vox, Voy, ts, i, x, y, Ytot, xtot, Ttot)
    
    Use Parametros
    Implicit None
    real :: Vox, Voy, a_rad, Vo !Entrada
    real :: Ttot, Ytot, Xtot !Salida
    real, dimension (0:puntos) :: x,y,ts !Internos
    integer :: i

    open (1, file = "SinArrastre.dat") 
!Calculos
   ts(0)=0
     
!Loop para cada delta de tiempo igual a 1 decima de segundo
    do i = 0, puntos, 1
      x(i) = vox*ts(i)
      y(i) = voy*ts(i) - .5*g*ts(i)*ts(i)
      ts(i+1) = (ts(i)+0.01)
        If (x(i)<0) then
          x(i)=0
        end if 
      write (1,1001) x(i), y(i)
     1001 format (f10.4, f10.4)
  !terminemos el loop cuando el objeto llegue al piso
        If (y(i)<0) exit
        end do   
   Close (1)
    Ttot = ts(i)
    Ytot = maxval (y, 1, (y(i)<0))
!Discriminemos valores cercanos a cero
    IF (a_rad == 0 ) THEN
      Xtot = 0
     Else  IF (a_rad == pi/2 ) THEN
      Xtot = 0  
     ELSE 
      Xtot = x(i) 
   END IF 
End Subroutine SinFriccion

!------------------------------------------------------------------------------

!Considerando la fuerza de arrastre del aire.

Subroutine ConArrastre (Vox, Voy, A_rad, Vo, CD, A, t, Vx, Vy, ax, ay,X1, 
Y1, Xtotal, Ytotal, Ttotal, Objeto)

    Use Parametros
    Implicit None
    real :: a_rad, Vo, Vox, Voy !Entrada
    real :: CD, A, m, D, r, h !Internos
    Character :: Objeto !Interno
    real, Dimension (0:puntos) :: t, Vx, Vy, Vin, ax, ay, X1, Y1 !Interno
    integer :: i
    real :: Xtotal, Ytotal, Ttotal !Salida


    write (*,*) "Selecciona la forma del objeto a lanzar:"
    write (*,*) "a=Esfera, b=Media esfera, c=cono, d=Cubo, e=Romboide, 
    f= Cilindro largo, g=Cililndro Corto"
    read *,objeto
    write (*,*) "Escriba el masa del objeto en kilogramos"
    read *, m

!Obtenemos el area transversa para cada forma de objeto y designamos su 
Coeficiente de friccion
Select case (objeto)
   Case ("a")
      Write (*,*) "Escriba el radio de la esfera"
      read *,r
      A = pi *r*r
      CD= 0.4700
   Case ("b")
      Write (*,*) "Escriba el radio de la media esfera"
      read *,r
      A = .5*pi*r*r
      CD = 0.42
   Case ("c")   
      Write (*,*) "Escriba el radio del cono"
      read *,r
      A=pi*r*r
      CD=0.50
   Case ("d")
      Write (*,*) "Escriba la medida de un lado del cubo"
      read *,h
      A= r*r
      CD=1.05
   Case ("f")
      Write (*,*) "Escriba la medida de un lado del rombo"
      read *,r
      A= r*sqrt(2*r*r)
      CD=0.80 
!Asumamos que los cilindros son de base cuadrada/rectangular
   Case ("g")
      Write (*,*) "Escriba el ancho de la base del cilindro"
      read *,h
      write (*,*) "Escriba el largo de la base del cilindro"
      read *,r
      A=r*h
      CD=0.82
   Case ("h")
      Write (*,*) "Escriba el ancho de la base del cilindro"
      read *,h
      write (*,*) "Escriba el largo de la base del cilindro"
      read *,r
      A=r*h
      CD=1.15
   Case Default
      Write (*,*) "Objeto no existente"
End Select
    
!Comenzamos la generacion de coordenadas
 
   Open (2, file = "ConArrastre.dat")

!Determinemos los valores iniciales
   X1(0) = 0
   Y1(0) = 0
   Vx(0) = Vox
   Vy(0) = Voy
   Vin(0)= Vo
   t(0) = 0
   D = 0.5*dAire*Cd*A
   ax(0) = -(D/m)*Vo*Vox
   ay(0) = -g-(D/m)*Vo*Voy
   
   write(2, 1002) X1(0), Y1(0)
   1002 format (f10.4,f10.4)

!XCalculemos el resto de las coordenadas

    Do i= 0, puntos, 1
      
     t(i+1) = t(i)+Delta
      Vx(i+1) = Vx(i)+ax(i)*t(i+1)
      Vy(i+1) = Vy(i)+ay(i)*t(i+1)
      X1(i+1) = X1(i)+Vx(i)*t(i+1)+(1/2)*ax(i)*t(i+1)*t(i+1)
      Y1(i+1) = Y1(i)+Vy(i)*t(i+1)+(1/2)*ay(i)*t(i+1)*t(i+1)
      ax(i+1)= -(D/m)*Vx(i)*Vx(i)
      ay(i+1)= -g-((D/m)*Vy(i)*Vy(i))
      If (y1(i)<0) exit
    Write (2,1003)  X1(i+1), Y1(i+1)
    1003  format (f10.4, f10.4 )

    End do
Close (2)
Xtotal = X1(i+1)
Ytotal = Maxval(Y1)
Ttotal = t(i)*10

End Subroutine ConArrastre
!-------------------------------------------------------------------------------------
\end{verbatim}
\pagebreak

\section{SALIDAS}
\subsection{Trayectorias para lanzamiento a 30 grados}
Se corrió el código para 30 grados con los siguientes resultados:\\
SALIDA\\
\includegraphics[scale=.50]{Salida30grados.png}
GRAFICA\\
\includegraphics[scale=.55]{Grafica30grad.png}
\pagebreak
\subsection{Trayectorias para lanzamiento a 45 grados}

Se corrió el código para 45 grados con los siguientes resultados:\\
SALIDA\\
\includegraphics[scale=.50]{Salida45grados.png}
GRAFICA\\
\includegraphics[scale=.55]{Grafica45grad.png}

\subsection{Trayectorias para lanzamiento a 60 grados}
Se corrió el código para 60 grados con los siguientes resultados:\\
SALIDA\\
\includegraphics[scale=.50]{Salida60grados.png}
GRAFICA\\
\includegraphics[scale=.55]{Grafica60grad.png}

% Nunca debe faltar esta ultima linea.
\end{document}